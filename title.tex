\begin{titlepage}
  \begin{center}
    {\LARGE Characterisation and Control of Linear Optical Quantum Computers}
\\[1.5cm]
    {\Large Nick Russell}\\[1.5cm]
    {\today}\\[12.0cm]
  \end{center}
  \begin{flushleft}
    {A dissertation submitted to the University of Bristol in accordance with
    the requirements for award of the degree of Doctor of Philosophy in the
    Faculty of Science}
  \end{flushleft}
  \begin{flushright}
    Word count: 19870
  \end{flushright}
\end{titlepage}

\begin{abstract}
Quantum technologies have the potential to disrupt almost every aspect of our
lives, the classic example being the prospect of a quantum computer breaking the
classical encryption on which so much of the current world relies. Perhaps
fortunately, the technical challenges involved in building such a machine are
formidable and it remains a long term goal. At present, less disruptive quantum
technologies such as quantum key distribution and quantum random number
generation are already making their way out of the lab and being pursued by
commercial enterprises.

Between these two extremes, there is potential for applications of quantum
technologies that will provide capabilities far beyond classical computers but
can be realised on a much shorter timescale. The \bosonsampling{} algorithm
could provide a theoretically rigorous example of quantum superiority, while
analogue quantum simulators demonstrate a more practically relevant task. In
this thesis I focus on these two fields, and progress towards realising them
in linear optics. I describe the use of reconfigurable linear optical circuits
to perform simulations of vibrational states of molecules, and extend the
simulation technique to open quantum systems.

In addition to the challenge of performing these computations, a lot of
work must go in to the related tasks of controlling the equipment and verifying
its output. This thesis discusses some of these procedures in detail, in
particular verification of correct operation of reconfigurable linear optical
circuits and control of a \bosonsampling{} experiment. Finally, I present a new
tomography procedure for experimentally determining the Hamiltonian underlying
evolution of a system of non-interacting bosons, and describe an experimental
realisation in a photonic quantum walk.
\end{abstract}

\begin{acknowledgements}
The Centre for Quantum Photonics at the University of Bristol has provided the
resources for me to conduct my PhD research, from office and lab space to
supervision and support. I'd like to thank Jeremy O'Brien for giving me the
opportunity to study for my PhD under his supervision, providing and
direction when necessary, and freedom to take my own path when appropriate.
Anthony Laing has been involved in almost all my research projects and I am
grateful for his support and mentoring through my four years at Bristol.

As the CQP has grown (and it's grown a lot since I started), the number of PhD
students has too. There isn't enough space to thank everyone by name but
there are a few fellow students who I would like to mention in 
particular. I worked closely with Enrique Mart\'in L\'opez from the very
beginning. He taught me most of my lab skills and gave me a lot of good advice
through all stages of my PhD. Jacques Carolan and I started in the CQP
at about the same time, and have often had similar interests. I've always found
our discussions enlightening and motivating, and our different approaches and
specialities have worked well together. My interactions with Chris Sparrow have
been similarly fruitful, from helping me out in the lab to helping correct my
theory.

Finally, I'd like to thank the administrative staff who really keep the
research group running. Rebecca Morton, Meagan Puzacke, Katharine Blackwell,
Kim Brooke and Caroline Clark have all been a great help to me at stages of
my studies, and I am very grateful for their support.

Outside the CQP, friends and family have contributed to my work and my life
outside of work. I'd particularly like to thank my parents and my partner Emily
for their interest in my research and, perhaps more importantly, knowing when
it's better not to ask!
\end{acknowledgements}

\begin{declaration}
  I declare that the work in this dissertation was carried out in accordance
  with the requirements of the University's Regulations and Code of Practice for
  Research Degree Programmes and that it has not been submitted for any other
  academic award. Except where indicated by specific reference in the text, the
  work is the candidate's own work. Work done in collaboration with, or with the
  assistance of, others, is indicated as such. Any views expressed in the
  dissertation are those of the author.

  SIGNED: \dotfill DATE: \dotfill
\end{declaration}

\tableofcontents
\listoffigures

\setstretch{1.5}
