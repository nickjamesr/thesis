\chapter{Tomography of Hamiltonians}
\label{ch:Hamiltomo}

\section{Introduction}
\label{sec:HTIntro}
The hamiltomo project. Using super-stable tomography, we can measure the unitary
description of a photonic device. If the unitary is generated by a time-constant
Hamiltonian (as in quantum walks) then successive measurements of \(\mat{U}\)
(at least 2) can be used to reconstruct the underlying Hamiltonian. Use the time
derivative method.

\section{Tomography of a unitary}
\label{sec:SST}
For background, I will present some of the methods used to determine fixed
unitaries. These methods are mostly not mine, but they make the point that there
are many ways to determine unitaries.

\section{Tomography of a Hamiltonian}
\label{sec:Hamiltomo}
This is where I will discuss the theory behind the time-derivative method,
including the step size considerations. This is all my own work.

\section{Experimental implementation}
\label{sec:HTExperiment}
Describe the methods (wavelength shift) and results of the experiment. Note that
Jacques did the experimental bits; I have done the analysis. Detail on the
effect of reference row/column choice can be here or in
section~\ref{sec:Hamiltomo}

