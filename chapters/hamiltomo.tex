\chapter{Tomography of Hamiltonians}
\label{ch:Hamiltomo}

\section{Introduction}
\label{sec:HTIntro}
In this chapter I discuss a procedure that I developed to determine the
(time constant) Hamiltonian governing the evolution of a system of
non-interacting Bosons in a discrete space. The process requires measurements of
single particle states and two particle correlations, and the number of
measurements scales polynomially with the number of system modes. It relies on
making measurements of the time evolution operator for multiple (minimum two,
preferably at least three) time increments.

In order to measure the time
evolution operator, I rely the method described in~\cite{sst} for efficiently
determining the unitary describing a linear optical circuit, or a similar
method. In section~\ref{sec:SST} I present a review of this method, and how it
is adapted to work in practice. Section~\ref{sec:Hamiltomo} describes in detail
the process for determining the Hamiltonian from multiple measurements of the
time evolution operator. Finally, section~\ref{sec:HTExperiment} contains
results of an experimental application of the process to a linear optical
quantum walk.

\section{Tomography of a unitary}
\label{sec:SST}
This section is a review of the method in~\cite{sst} for efficiently determining
the unitary description of a linear optical circuit. It is not my own work, but
it is important background for the results that follow.

\section{Tomography of a Hamiltonian}
\label{sec:Hamiltomo}
Now that we know we can determine the unitary description of a fixed quantum
process in linear optics, I apply this result to a general system of
non-interacting bosons evolving under a time-constant Hamiltonian in a discrete,
finite space. In particular, using measurements of the unitary time evolution
operator for the system at multiple time increments, we can deduce the
Hamiltonian underlying the evolution.

In practice, the experimental requirements are:
\begin{itemize}
  \item Preparation of arbitrary 1- and 2-particle states in the computational
  basis
  \item Precise control over the evolution time
  \item Arbitrary measurement of mode occupation and pairwise correlations
  between occupations (again, in the computational basis)
\end{itemize}

Recall that in a discrete, finite space of \(m\) orthogonal modes, the
Hamiltonian can be expressed exactly as an \(m \by m\) Hermitian matrix, and the
time evolution operator is an \(m \by m\) unitary matrix. If the Hamiltonian
\(\mat{H}\) is constant in time, the relationship between the time evolution
operator, \(\mat{U}\) and the Hamiltonian is:
\begin{equation}
  \label{eq:exponential}
  \mat{U} \of{t} = e^{-i \mat{H} t}
\end{equation}

Now consider taking the time derivative of this expression:
\begin{equation}
  \mat{U}^{\prime} \of{t} = -i \mat{H} e^{-i \mat{H} t} = -i \mat{H} \mat{U}
  \of{t}
\end{equation}
If we could measure \(\mat{U}^{\prime}\) directly, we would be done at this
point and the Hamiltonian would be revealed by:
\begin{equation}
  \mat{H} = i \mat{U}^{\prime} \of{t} \mat{U}^{\dagger} \of{t}
\end{equation}
for any time \(t\).

In order to avoid reliance on direct measurements of \(\mat{U}^{\prime}\), we
can instead discretise the derivative, and take multiple measurements of
\(\mat{U}\) in place of a single measurement of \(\mat{U}^{\prime}\). Here I
present the commonly-used 3-point time derivative, and derive the following
expression for \(\mat{H}\):
\begin{equation}
  \mat{H} = \frac{i}{2 \dt} \left[ \mat{U} \of{t+\dt} - \mat{U} \of{t-\dt}
  \right] \mat{U}^{\dagger} \of{t}
\end{equation}
Note that this is not the only way to recover the Hamiltonian from measurements
of \(\mat{U}\). Alternatives, such as the matrix logarithm and series expansions
are discussed briefly in section~\ref{sec:Alternatives}. The rest of this
section concerns the factors involved in choosing the step size, \(\dt\).

\section{Experimental implementation}
\label{sec:HTExperiment}
In photonics we have excellent sources and detectors, so the requirements of
arbitrary preparation and measurement in the computational basis are easy to
meet. The requirement that we have control over the evolution time is a more
difficult. In free space we could move the detectors closer to or further from
the source, but the Hamiltonian describing photons propagating in free space is
diagonal so the experiment would not be very interesting. Free space components
with more interesting Hamiltonians such as waveplates tend to be fixed, so we
have no control whatsoever over the evolution time!

A solution is presented in integrated optics. Consider a continuously coupled
quantum walk, where each waveguide is coupled to its nearest neighbour: the
effective evolution time is proportional to the optical path length of the
coupling region. We cannot easily vary the physical length of the coupling
region, but by varying the wavelength of the light, we can change the optical
length. It is using this technique that we perform the experiment described in
this section. Note that the setup and wavelength shifting of the source, and
most of the data acquisition was was done by PhD student Jacques Carolan. I did
the remaining data acquisition and all the data analysis.

\section{Alternative methods}
\label{sec:Alternatives}
On inspection of equation~\ref{eq:exponential}, the most straightforward way to
recover the Hamiltonian from a measurement of \(\mat{U}\) would appear to be to
take the logarithm:
\begin{equation}
  \mat{H} = \frac{i}{t} \log \of{\mat{U} \of{t}}
\end{equation}
This can be performed either by diagonalising \(\mat{U}\) or by taking a series
expansion, both of which have their drawbacks. When diagonalising a unitary
matrix, all eigenvalues are unit-modulus complex numbers which permit a freedom
of \(\pm 2 n \pi\) when taking the logarithm. Therefore the resulting
Hamiltonian is not unique. The series expansion has poor convergence properties,
so only small time steps may be considered and many terms may be required to
reach convergence. Experimentally, if there are errors in the measurement of
\(\mat{U}\), taking high order terms causes a multiplication of these errors,
which puts a practical limit on the number of terms that may be used. I do note
that these considerations are not dissimilar to the step size considerations
that I discuss above.

An improved series-type solution has been described in~\cite{hamiltomo}, where
the Hermitian and anti-Hermitian parts of \(\mat{U}\) are considered separately.





