\chapter{Conclusions}
\label{ch:Conclusions}
In this thesis I have discussed the use of non-interacting bosons (photons) to
implement non-universal models of quantum computation. I have presented
proof-of-principle experiments showing that useful quantum simulations can be
performed on this physical platform without the overheads of a universal quantum
computer, and I have developed a series of tools for verifying the results of
such systems.

The simulation experiments that I presented in chapter~\ref{ch:Simulations} open
up new and exciting possibilities of performing quantum simulations without the
massive technical overheads of a universal quantum computer. I have demonstrate
that the scope of these simulations extends beyond the isolated quantum systems
usually considered in quantum mechanics, to open quantum systems with very
different dynamics. The immediate application to vibrational states of molecules
suggests that this is set to soon become a useful simulation technique for
chemists, and that other applications will probably be discovered.

I believe that the verification and control procedures presented in other
chapters will become important, even essential in the development of
intermediate models of quantum computation. In the realm of \bosonsampling{},
the scale of experiments will grow with the capabilities of single photon
sources, which will potentially see a sharp increase when multiplexing
technologies mature. At this point, methods such as the dialling procedure in
chapter~\ref{ch:DirectDialling} and the verification techniques presented in
chapter~\ref{ch:QCV} will become invaluable. It is not clear whether
verification in terms of clouding will be widely deployed, but its mere
existence (along with the existence of other procedures such as the zero
transmission law) is important evidence that we will be able to gain confidence
in the correct operation of large quantum devices. 

Finally, the tomography protocol presented in chapter~\ref{ch:Hamiltomo} is
another useful tool for understanding systems of non-interacting bosons.
Tomography is the ultimate proof of correct operation of a device, but general
quantum process tomography has the unfortunate property of scaling exponentially
with the size of the physical system. This new protocol takes advantage of
additional assumptions about the system (no interactions and piecewise constant
Hamiltonian) in order to obtain favourable scaling. Irrespective of whether any
of these procedures are widely adopted in the form presented here, this idea of
taking advantage of properties of the physical system is very generally
applicable. In the context of non-universal quantum computing, I expect it to be
the single most important driver of progress.

Based on the current state of non-universal quantum computing in linear optics,
I believe it has potential to be the next quantum technology to become widely
deployed. I am confident that we will start seeing useful applications of
analogue quantum simulators long before digital quantum simulation becomes
feasible. The experiments in this thesis, and the verification and control
procedures represent an important step towards achieving this goal.
