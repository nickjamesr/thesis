\chapter{Boson simulations (it's just diagonalisation)}
\label{ch:Simulations}

\section{Introduction}
This formed the bulk of the experimental work in my PhD. Roughly speaking, we
take a hamiltonian, \(\mat{H}\) and exponentiate it to get a load of unitaries
(\(\mat{U}=e^{-i\mat{H}t}\)) at a series of timesteps. Using a decomposition
similar to the one in \cite{reck94} we can implement these on a single
circuit.

Two physical implementations: bulk optics and integrated optics. I was
responsible for the calibration of the bulk circuit (crapusoids), writing the
control code, and taking a large portion of the data. I trained DTC student
Chris Sparrow in the operation of the experiment.

\section{Molecular vibrations}
Simulating phonons in the harmonic limit corresponds to vibrations of molecules.

\section{PT-symmetric systems}
Taking a non-Hermitian hamiltonian, \(\mat{G}\) corresponds to an open system
with loss and/or gain. A system with balanced but spatially separated loss and
gain is \pt-symmetric. Exponentiating this hamiltonian gives us a non-unitary
time-evolution operator, \(\mat{V}\). Using unitary dilation, we implement this
as a sub-matrix of a unitary operator. Physically, this corresponds to coupling
the system to an environment with an equal number of spatial modes.
