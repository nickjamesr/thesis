\chapter{Direct Dialling}
\label{ch:DirectDialling}
\section{Introduction}
This chapter is concerned with universal circuits for linear optics, of which
the bulk optics circuit described in chapter~\ref{ch:QCVV} is a small
example.
The motivating question of the chapter is, if one wants to use a Reck
scheme\cite{reck94} to dial a Haar random unitary matrix, how should the
parameters be chosen? A priori, it is not clear that the reflectivities and
phase shifts in the Reck scheme will follow simple distributions, nor that they
can be chosen independently of each other.

I will begin by reviewing the Reck scheme, and how it may be implemented in
integrated optics. I then return to the question of dialling a Haar unitary on
this device, by way of a coordinate transformation to the physical parameters on
the device.

\section{A universal circuit for linear optics}
Any lossless\footnote{In practical devices, this approximation is very good.
What is the loss in dB/cm for our Reck scheme?} linear optical device can be
described by a unitary matrix operating on the optical modes. A device with
\(m\) input and \(m\) output modes will be described by an \(m \by m\) unitary.
In~\cite{reck94}, it is shown that the converse is true: any unitary matrix can
be realised by a linear optical circuit. A constructive proof of this is
presented, and a candidate `universal circuit' is proposed. Universal, in this
context refers to a linear optical circuit with adjustable components
(beamsplitters with variable reflectivity and variable phase shifts), which can
be configured into any unitary.

\section{Arithmetic of probability density functions}
I spent a lot of time considering these. I think I can present this as
preliminary material.

\section{The Reck scheme}
More background. Essential for several sections/chapters.

\section{Results: the distributions}
Explicitly lay out what probability distributions are required.

\section{Proof}
In terms of a coordinate transformation

\section{Applications}
Most obviously, generating Haar unitaries (or truncations thereof) directly in
experimental parameters

Further work: reconstructing a unitary using Bayesian inference on this
parameterisation.
