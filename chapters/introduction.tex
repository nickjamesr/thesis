\chapter{Introduction}
\label{ch:Introduction}
\section{General Introduction}
\label{sec:Science}
The race for the first quantum computer is on. Academic researchers and private
enterprises \cite{dwave} around the world are pursuing this goal, on a number of
different physical platforms. Here, I will describe my efforts towards achieving
quantum computation in linear optics.

Of course, building a full-scale, universal quantum computer is going to be
difficult, and beyond the scope of any single PhD student's research. At the
other end of the spectrum, some quantum technologies have already found
commercial viability \cite{idquantique,magiq,qutools}, putting them beyond 
the realms of academic research. I will focus on the middle ground between
these two extremes: quantum computers that fall short of being universal for
BQP, yet may be able to outperform a classical computer at certain, specific tasks in the near future.

One such device would be the Boson Sampler. Proposed by Aaronson and Arkhipov
in~\cite{bosonsampling}, this is a device somewhat native to linear optics.
While it is
a good candidate for beating a classical computer, its uses are somewhat limited
at the moment. I will discuss two results relevant to building a Boson Sampler:
in chapter~\ref{ch:DirectDialling} I present a theoretical result concerning
Haar random unitary matrices; chapter~\ref{ch:QCV} presents a demonstration of
practical techniques to verify the output of such a device. Experimental results
presented in chapter~\ref{ch:QCV} could be said to be implementing
\bosonsampling{} on a small scale.

The second class of quantum computer that I shall discuss is the quantum
simulator. The purpose of quantum simulation is to use a device that is well
understood and controlled, in order to simulate the behaviour of a quantum
system that is less accessible. In chapter~\ref{ch:Simulations} I present
experimental results of quantum simulations using linear optics. Two systems are
simulated: the (quantized) vibrations of molecules, and more abstract systems of
bosons obeying \(\pt\)-symmetric (but non-Hermitian) dynamics.

Once we begin to scale up the size of these devices, it will be essential to
have procedures to test whether they are working correctly. Since the original
point is to outperform a classical computer, they cannot simply be checked by
performing the same calculations classically! In the case of problems in NP such
as Shor factoring, we have efficient methods to check the output, but this is
not guaranteed for the problems discussed here and the issue of verification in
\bosonsampling is particularly important. Chapter~\ref{ch:QCV} describes methods
for the calibration and verification of quantum optics devices. I present a
detailed case study of the calibration of a circuit in bulk optics, which
illustrates many of the complicating features present in any large optical
device. A series of verification procedures are applied to systems of up to 5
photons in a quantum walk and 6 photons in a quantum Fourier transform.

Finally, in chapter~\ref{ch:Hamiltomo} I describe a tomography protocol that can
be used in certain circumstances to efficiently determine the Hamiltonian
describing a dynamical quantum system. Tomography arguably provides the ultimate
verification of correct operation, but quantum process tomography has the
unfortunate property of scaling exponentially with the size of the system. The
procedure presented here applies when we can make simplifying assumptions,
namely that the constituent particles of the system do not interact and that the
Hamiltonian is constant (or at least piecewise constant) in time. I present an
experimental demonstration of the protocol on a photonic quantum walk, but it
could have applications outside of quantum photonics.o

\section{Organisation of this Thesis}
\label{sec:Organisation}
Every chapter opens with a few paragraphs attributing the work that contributed
to the chapter. \todo{Say something explaining that it's a big, collaborative
field/research group, and of course I didn't do all the work on my own.}

