\chapter{Introduction}
\label{ch:Introduction}
\section{Scientific matters}
\label{sec:science}
The race for the first quantum computer is on. Academic researchers and private
enterprises around the world are pursuing this goal, on a number of different
physical platforms. Here, I will describe my efforts towards achieving quantum
computation in linear optics.

Of course, building a full-scale, universal quantum computer is going to be
difficult, and beyond the scope of any single PhD student's research. At the
other end of the spectrum, some quantum technologies have already found
commercial viability, also putting them outside of the realms of academic
research. I will focus on the middle ground between these two extremes: quantum
computers that fall short of being universal for BQP, yet may be able to
outperform a classical computer at certain, specific tasks in the near future.

One such device would be the Boson Sampler. Proposed by Aaronson and Arkhipov
in~\cite{bosonsampling}, this is a device somewhat native to linear optics.
While it is
a good candidate for beating a classical computer, its uses are somewhat limited
at the moment. I will discuss two results relevant to building a Boson Sampler:
in chapter~\ref{ch:DirectDialling} I present a theoretical result concerning
Haar random unitary matrices; chapter~\ref{ch:QCVV} presents a demonstration of
practical techniques to verify the output of such a device. Experimental results
presented in chapter~\ref{ch:QCVV} could be said to be implementing
\bosonsampling{ }on a small scale.

The second class of quantum computer that I shall discuss is the quantum
simulator. The purpose of quantum simulation is to use a device that is well
understood and controlled, in order to simulate the behaviour of a quantum
system that is less accessible. In chapter~\ref{ch:Simulations} I present
experimental results of quantum simulations using linear optics. Two systems are
simulated: the (quantized) vibrations of molecules, and more abstract systems of
bosons obeying \pt-symmetric (but non-Hermitian) dynamics.

Once these devices begin to become larger, it is essential to have procedures to
test whether they are working correctly. Since the original point is to
outperform a classical computer, they cannot simply be checked by performing the
same calculations classically! Chapter~\ref{ch:QCVV} presents methods for the
calibration, validation and verification of quantum devices. The calibration
procedure is described in detail for a circuit in bulk optics, which illustrates
many of the complicating features present in any large optical device. The
verification procedure is used to check the coherence \todo{how do I say this? I
mean indistinguishability of photons} of up to 5 photons in a quantum walk.

Finally, chapter~\ref{ch:Hamiltomo} describes a tomography protocol that can be
used in certain circumstances to arrive at a complete description of the device
in question.
