\documentclass[a4paper,11pt]{article}
\usepackage{thesis}

\begin{document}

\section{}
\begin{quote}
  ``Please correct the places where `distance' is used incorrectly.''
\end{quote}
The paragraph defining the fidelity has been rewritten as follows:
\begin{quote}
  ``To evaluate the fidelity of our simulations, we look at two quantities. The
  first is computed from the trace of the product of the unitary matrix that we
  intended to implement, and the matrix that we in fact did implement, according
  to the data we collect. If the desired unitary is \(\mat{U}\) and
  the measured unitary is \(\mat{U^{\prime}}\), this fidelity,
  \(F_{\mathrm{tr}}\) is defined by:
  \begin{equation}
    F_{\mathrm{tr}}=\frac{1}{2}\abs{\Tr\of{\mat{U}^{\dagger}{\mat{U}^{\prime}}}}
  \end{equation}
  In the case of perfect implementation, \(\mat{U}\) and \(\mat{U^{\prime}}\)
  will be the same (and both unitary), yielding unit fidelity; an imperfect
  implementation results in a fidelity less than one.''
\end{quote}
Previously incorrect uses of `distance' throughout the document now reference
these `fidelities' and make no claim to be a `distance'.

\section{}
\begin{quote}
  ``Please consider the relative impact of changes in mode index vs
  mode coupling as a function of wavelength on p100 and provide a
  revised paragraph.''
\end{quote}
The relative impact of the two effects has been considered and I have come to
the conclusion that the change in mode index is likely to be greater than the
change in coupling strength. I propose the following revision to the paragraph
in question:
\begin{quote}
  ``The experimental results in this chapter require measurements of one and two
  photon states after evolution for three different times in a continuous time
  quantum walk. The effective evolution time in a linear optical quantum walk is
  dictated by the length of the coupling region, the effective index of
  refraction for guided light, and the strength of coupling between adjacent
  waveguides. Since we cannot change the length of the coupling region in a
  monolithic, integrated device, we must instead rely on the other two effects.
  Both the mode index and the coupling strength are a function of the wavelength
  of the photons, so by using a variable wavelength source, we can achieve
  different evolution times with the same chip.

  Tuning the wavelength does not give us independent control over the mode index
  and the coupling strength, so the effect on evolution time,
  \(t_{\text{eff}}\), will necessarily be a compination of these two factors.
  The couplings, \(\eta_{ij}\), appear on the off-diagonals of the Hamiltonian,
  \(\mat{H}\), which only appears in our results as the product \(\mat{H} t\).
  Hence, any transformation of the form \(\eta_{ij} \rightarrow k \eta_{ij}\)
  for some constant \(k\) will have exactly the same effect as the
  transformation \(t \rightarrow k t\). For a small change in wavelength, we
  can approximate the change in mode index as linear, \(n \of{\lambda} = \bigo
  \of{\lambda}\). Since the couplings depend on the overlap of the evanescent
  field, the dependence on wavelength is exponential and consequently we expect
  the effect of the change in coupling strength to be smaller than the effect of
  the change in mode index.''
\end{quote}

\section{}
\begin{quote}
  ``Please give an explicit description of how the parametrisation of
  Haar random unitaries given in the thesis produces the correct
  distribution of elements of the first column of the unitary in the 3x3
  case.  Please also use the parametrisation in the 3x3 case to produce
  a number of random unitaries and confirm that they are correctly
  distributed [using some appropriate measure of your choice].''
\end{quote}

\section{}
\begin{quote}
  ``Please update the Appendix to correctly list the preprints as
  referred to elsewhere in the thesis.''
\end{quote}
I have inserted the following text into the appendix, after the list of
first-author preprints:
\begin{quote}
  ``I am co-author on one additional preprint
  \begin{itemize}
    \item E. Mart\'in L\'opez, N. J. Russell, C. Sparrow, J. L. O'Brien and A.
      Laing, ``Quantum simulation of phonons in the harmonic limit'',
      \textit{in preparation}.''
  \end{itemize}
\end{quote}

\section{}
\begin{quote}
  ``Please either produce a revised Figure 4.6 or find another way to
  make sure that this is correctly described''
\end{quote}
Figure 4.6 has been revised to show the correct axis labels. The new version
(figure \ref{fig:manifolds}) is reproduced here.
\begin{figure}
  \centering
  \includegraphics{manifolds}
  \caption{Revised version of figure 4.6, with the axes correctly labelled.}
  \label{fig:manifolds}
\end{figure}

\section{}
\subsection*{Other minor edits made}
\begin{enumerate}
  \item Add number to the three equations formerly missing them.
  \item Assign different symbols to the figures of merit in section 4.3.2.
  \item Correct other minor typographical errors (e.g. spelling and
        punctuation).
\end{enumerate}

\end{document}
