
\documentclass[11pt,a4paper,twoside]{report}

\newcommand{\bosonsampling}{\textsc{BosonSampling}}
\newcommand{\rstar}{R^{*}}
\newcommand{\pt}{\(\mathcal{PT}\)}
\newcommand{\mat}[1]{\mathrm{#1}}

\begin{document}

\chapter{Indroduction}

Outline some kind of logical structure.

\chapter{Direct Dialling}

Early theoretical work that I did. Motivated by the question: if we want a Haar
unitary from a Reck scheme\cite{reck94}, can we just set the parameters
(reflectivities and phase shifts) randomly and independenly? If so, what
distributions? Answer: yes and a load of beta distributions, depending on where
the beamsplitter is.

\section{Arithmetic of probability density functions}
I spent a lot of time considering these. I think I can present this as
preliminary material.

\section{The Reck scheme}
More background. Essential for several sections/chapters.

\section{Results: the distributions}
Explicitly lay out what probability distributions are required.

\section{Proof}
In terms of a coordinate transformation

\section{Applications}
Most obviously, generating Haar unitaries (or truncations thereof) directly in
experimental parameters

Further work: reconstructing a unitary using Bayesian inference on this
parameterisation.

\chapter{Boson simulations}

This formed the bulk of the experimental work in my PhD. Roughly speaking, we
take a hamiltonian, \(\mat{H}\) and exponentiate it to get a load of unitaries
(\(\mat{U}=e^{-i\mat{H}t}\)) at a series of timesteps. Using a decomposition
similar to the one in \cite{reck94} we can implement these on a single
circuit.

Two physical implementations: bulk optics and integrated optics. I was
responsible for the calibration of the bulk circuit (crapusoids), writing the
control code, and taking a large portion of the data. I trained DTC student
Chris Sparrow in the operation of the experiment.

\section{Molecular vibrations}
Simulating phonons in the harmonic limit corresponds to vibrations of molecules.

\section{\pt-symmetric systems}
Taking a non-Hermitian hamiltonian, \(\mat{G}\) corresponds to an open system
with loss and/or gain. A system with balanced but spatially separated loss and
gain is \pt-symmetric. Exponentiating this hamiltonian gives us a non-unitary
time-evolution operator, \(\mat{V}\). Using unitary dilation, we implement this
as a sub-matrix of a unitary operator. Physically, this correspods to coupling
the system to an environment with an equal number of spatial modes.

\chapter{Verification}

Work relating to verification in quantum walks and \bosonsampling. I don't think
it's unfair to say that I had the original motivation to implement the
\(\rstar\) protocol described in \cite{aaronson13}. I did the Bayesian model
comparison between \bosonsampling and uniform distributions.

\chapter{Tomography of Hamiltonians}
The hamiltomo project. Using super-stable tomography, we can measure the unitary
description of a photonic device. If the unitary is generated by a time-constant
Hamiltonian (as in quantum walks) then successive measurements of \(\mat{U}\)
(at least 2) can be used to reconstruct the underlying Hamiltonian. Use the time
derivative method.

\chapter{Decoupling}

Project with Pete Turner.

\bibliographystyle{abbrv}
\bibliography{bib/thesis}

\end{document}
